En la presente documentación, se documentarán los métodos y clases implementadas en la metaheurística C.\+C.\+O. ({\bfseries C}lonal {\bfseries C}olony {\bfseries O}ptmization), que se encuentran codificados en el lenguaje de programación {\bfseries C++}.

El principal objetivo que se desea lograr al implementar la metaheurística adaptativa e inspirada en la naturaleza C.\+C.\+O., es la de hallar soluciones robustas en problemas de optimización, sin incurrir en costos computacionales extras al evaluar la robustez de las soluciones candidatas. Ésto es posible, al imitar el esquema colaborativo de la {\bfseries colonias clonales} presente en la naturaleza.

La estructura básica de la metaheurística C.\+C.\+O., se encuentra constituida por un {\bfseries algoritmo iterativo generacional}, el cual se describe a continuación\+:



Un complemento idóneo a la presente documentación, correspondería a un reporte técnico. La utilidad que nos presentaría un reporte técnico, sería la de permitirnos relacionar a el código implementado, con la lógica de la metaheurística C.\+C.\+O. 
\begin{DoxyItemize}
\item Enlace disponible para el reporte técnico\+: \href{../html2/HTLATEXDocumentacionCCO.html}{\tt Link-\/$>$}  
\end{DoxyItemize}

Por último, esto es parte de un trabajo de tesis realizado para optar al título de Ingeniero Civil en Informática de la \href{www.uach.cl}{\tt Universidad Austral de Chile}, por lo que las personas relacionadas con ésta, son las siguientes\+:

{\bfseries Profesor patrocinante\+:} 
\begin{DoxyItemize}
\item Dr. Ing. Jorge Patricio Maturana Ortíz (\href{http://www.inf.uach.cl/maturana/}{\tt Personal Page-\/$>$})  
\end{DoxyItemize}{\bfseries Alumno tesista\+:} 
\begin{DoxyItemize}
\item Mauricio Alexis Guarda Oñate 
\end{DoxyItemize}